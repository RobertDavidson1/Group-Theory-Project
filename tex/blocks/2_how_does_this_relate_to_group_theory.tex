From a group theory perspective, we see that our random walker walks on the set of integers $\mathbb{Z}$, equipped with the group operation of addition, and forms the infinite cyclic abelian group:
$$(\mathbb{Z}, +)$$ 
The group satisfies the group axioms, proven in Lecture 1.
Every element of $\mathbb{Z}$ can be generated by repeatedly applying the group operation with the element $1$:
$$\mathbb{Z} = \langle 1 \rangle = \{ k\cdot 1 : k \in \mathbb{Z} \}$$
So, $1$ is a generator of the group, and its inverse, $-1$, also generates the group. 

Thus the set: 
$$\langle S \rangle = \langle \{1,-1\} \rangle = \mathbb{Z}$$ 
forms a generating set for $(\mathbb{Z}, +).$ 

We can describe our random walk entirely in group theory notation, starting at the identity element ($x = e$), each step corresponds to performing the group operation with some element from the generating set, and so after $t$ steps:
$$P_t = e+ s_1 + s_2 + \dots + s_t,  \quad s_i \in S$$
Since $x = e = 0$, rewriting gives:
$$P_t =  \sum_{i=1}^t s_i  \quad \textrm{where $s_i \in S$ is chosen uniformly at random}$$