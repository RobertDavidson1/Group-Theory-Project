In this project we studied simple random walks through the perspective of Group Theory.  We started by introducing a random walk on the integers, and viewed it as  the walk on the group $(\mathbb{Z}, +)$. We then restricted the walk to $\mathbb{Z}_n$ and discussed why the subgroup of even residues forces the walk to alternate between two disjoint cosets, and prevents use from reaching a uniform distribution for the final positions of our walkers. We then extended this to the direct product $\mathbb{Z}_n \times \mathbb{Z}_n$ and saw a similar restriction. 

Random walks on groups appear in everyday examples. Shuffling cards is essentially taking random steps through the group, $S_{52}$ all possible orders of the deck\cite{sahlsten_cards}. Scrambling a Rubik's Cube works the same way, where it corresponds to a random walk on a group containing 43 quintillion elements\cite{qu_rubiks}. Some cryptographic methods also use repeated random steps through a group to search for hidden information\cite{nolan_crypto}.